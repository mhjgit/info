%%
%% Automatically generated file from DocOnce source
%% (https://github.com/hplgit/doconce/)
%%
%%


%-------------------- begin preamble ----------------------

\documentclass[%
oneside,                 % oneside: electronic viewing, twoside: printing
final,                   % draft: marks overfull hboxes, figures with paths
10pt]{article}

\listfiles               %  print all files needed to compile this document

\usepackage{relsize,makeidx,color,setspace,amsmath,amsfonts,amssymb}
\usepackage[table]{xcolor}
\usepackage{bm,ltablex,microtype}

\usepackage[pdftex]{graphicx}

\usepackage[T1]{fontenc}
%\usepackage[latin1]{inputenc}
\usepackage{ucs}
\usepackage[utf8x]{inputenc}

\usepackage{lmodern}         % Latin Modern fonts derived from Computer Modern

% Hyperlinks in PDF:
\definecolor{linkcolor}{rgb}{0,0,0.4}
\usepackage{hyperref}
\hypersetup{
    breaklinks=true,
    colorlinks=true,
    linkcolor=linkcolor,
    urlcolor=linkcolor,
    citecolor=black,
    filecolor=black,
    %filecolor=blue,
    pdfmenubar=true,
    pdftoolbar=true,
    bookmarksdepth=3   % Uncomment (and tweak) for PDF bookmarks with more levels than the TOC
    }
%\hyperbaseurl{}   % hyperlinks are relative to this root

\setcounter{tocdepth}{2}  % levels in table of contents

% prevent orhpans and widows
\clubpenalty = 10000
\widowpenalty = 10000

% --- end of standard preamble for documents ---


% insert custom LaTeX commands...

\raggedbottom
\makeindex
\usepackage[totoc]{idxlayout}   % for index in the toc
\usepackage[nottoc]{tocbibind}  % for references/bibliography in the toc

%-------------------- end preamble ----------------------

\begin{document}

% matching end for #ifdef PREAMBLE

\newcommand{\exercisesection}[1]{\subsection*{#1}}


% ------------------- main content ----------------------



% ----------------- title -------------------------

\thispagestyle{empty}

\begin{center}
{\LARGE\bf
\begin{spacing}{1.25}
Søknad om Meritteringsordning for utdanningsfaglig kompetanse ved Universitetet i Oslo
\end{spacing}
}
\end{center}

% ----------------- author(s) -------------------------

\begin{center}
{\bf Morten Hjorth-Jensen${}^{1, 2}$} \\ [0mm]
\end{center}

\begin{center}
% List of all institutions:
\centerline{{\small ${}^1$Department of Physics and Center of Computing in Science Education, University of Oslo, Norway}}
\centerline{{\small ${}^2$Department of Physics and Astronomy and Facility of Rare Ion Beams and National Superconducting Cyclotron Laboratory, Michigan State University, USA}}
\end{center}
    
% ----------------- end author(s) -------------------------

% --- begin date ---
\begin{center}
14 Mai, 2020
\end{center}
% --- end date ---

\vspace{1cm}


\paragraph{Bøker:}
\begin{enumerate}
\item Morten Hjorth-Jensen, \emph{Computational Physics, an introduction}, to be published by IOP in 2020, 500 pages.

\item Morten Hjorth-Jensen, \emph{Computational Physics, an advanced course}, to be published by IOP in 2020, 400 pages

\item Morten Hjorth-Jensen, \emph{Nuclear many-body physics, a computational perspective}, in preparation for Lecture Notes in Physics by Springer.

\item \href{{http://www.springer.com/us/book/9783319533353}}{Morten Hjorth-Jensen, M.P. Lombardo and U. van Kolck}, \emph{Computational Nuclear Physics-Bridging the scales, from quarks to neutron stars}, Lectures Notes in Physics by Springer, Volume \textbf{936} (2017).
\end{enumerate}

\noindent
\paragraph{Artikler.}
\begin{enumerate}
\item John M. Aiken, Riccardo De Bin, Morten Hjorth-Jensen, Marcos D. Caballero, \emph{Predicting time to graduation at a large enrollment American university}, arXiv:2005.05104 

\item Marcos Daniel Caballero, Morten Hjorth-Jensen, \emph{Integrating a Computational Perspective in Physics Courses}, arXiv:1802.08871

\item Malthe-Sørenssen, Anders; Hjorth-Jensen, Morten; Langtangen, Hans Petter; Mørken, Knut Martin. \emph{Integrasjon av beregninger i fysikkundervisningen}, UNIPED, 38:303, 2015.
\end{enumerate}

\noindent

% ------------------- end of main content ---------------

\end{document}

