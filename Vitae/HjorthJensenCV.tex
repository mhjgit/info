\documentclass[11pt]{revtex4-1}
\usepackage{graphicx}
\usepackage{dcolumn}
\usepackage{bm}
\usepackage[colorlinks=true,urlcolor=blue,citecolor=blue]{hyperref}
\usepackage{color}

%\usepackage[square,comma,sort&compress]{natbib}
\usepackage{indentfirst}
\usepackage{hyperref}
\usepackage{pst-plot}
\usepackage{natbib}
\usepackage{mathrsfs}
\usepackage{epic,eepic}
\usepackage{amsfonts}
\usepackage{amsmath,amssymb}
\usepackage{enumitem}
\usepackage{subcaption}
\newcommand{\sssubsection}[1]{{\it #1}}
\newcommand{\kf}{k_{\scriptscriptstyle\rm F}}
\newcommand{\collab}[1]{{\small[\it #1\/}]}
\newcommand{\lsim}{\, \, \raisebox{-0.8ex}{$\stackrel{\textstyle <}{\sim}$ }}
% set up paper size
\setlength{\textwidth}{16.59cm}
\setlength{\marginparsep}{0pt}
\setlength{\marginparwidth}{0pt}
\setlength{\textheight}{22.94cm}
\setlength{\headheight}{0pt}
\setlength{\headsep}{0pt}
\setlength{\oddsidemargin}{-0.04cm}
\setlength{\topmargin}{-.04cm}
\renewcommand{\baselinestretch}{1.1}

\begin{document}

\section*{Vitae for Hjorth-Jensen, Morten}

\subsection*{Role in project:}
Principal Investigator

\subsection*{Personal data:}
\begin{itemize}
\item Name: Hjorth-Jensen, Morten, male
\item Norwegian citizen, born in Haugesund, Norway, permanent resident of the U.S.A.
\item Website \url{http://mhjgit.github.io/info/doc/web/}. Full Vitae can be downloaded from \url{http://mhjgit.github.io/info/doc/pub/cv/html/cv.html}. See Google Scholar \url{https://scholar.google.com/citations?user=nuiyEmwAAAAJ&hl=no} for citation metrics.

\end{itemize}

\subsection*{Professional preparation and education:}
\begin{itemize}
\item Norwegian University of Science and Technology, Trondheim, Norway,  Siv.Ing. in Theoretical Physics (Master of Science equivalent),  1988 

\item University of Oslo, Norway,  Ph.D in Theoretical Nuclear Physics, 1993

\item ECT*, Trento, Italy,  Postdoctoral Researcher in Theoretical Nuclear Physics,  1994-1996

\item Nordita, Copenhagen, Denmark, Postdoctoral Researcher in Theoretical Nuclear Physics, 1996-1998
\end{itemize}

\subsection*{Positions:}

Professor of Physics at Michigan State University, USA and the University of Oslo, Norway
\begin{quote}
\begin{tabular}{|l|l|l|}
\hline
\multicolumn{1}{|c}{ Position } & \multicolumn{1}{|c}{ Institution } & \multicolumn{1}{|c|}{ Dates } \\
\hline
Associate Professor of Physics & University of Oslo        & 1999-2001    \\
Professor of Physics           & University of Oslo        & 2001-present \\
Adjunct Professor of Physics   & Michigan State University & 2003-2011    \\
Professor of Physics           & Michigan State University & 2012-present \\
\hline
\end{tabular}
\end{quote}




\subsection*{Project management experience, last ten years}
\begin{enumerate}

\item 2020-2022 750 kUSD from the Department of Energy, From Quarks to Stars; A Quantum Computing Approach to the Nuclear Many-Body Problem. PI, grant number DoE-0000248785  

\item 2020-2023 600 kUSD from the National Science Foundation for the project From nuclei to neutron stars, PI with Scott Bogner, Grant number PHY-013877. 

\item 2016-present Co-PI  at the Norwegian center of excellence in Education \emph{Center for Computing in Science Education}, University of Oslo with annual funding from DIKU of 5MNOK

\item 2017-2020 600 kUSD from the National Science Foundation for the project From nuclei to neutron stars, PI with Scott Bogner. Grant number PHY-1713901

\item 2014-2017 600 kUSD from the National Science Foundation for the project Computational Nuclear Many-body Physics, PI with Scott Bogner. Grant number PHY-1404159

\item 2010-2015 15 MNOK from the Research Council of Norway, Multi-scale physics on the computer, Collaboration between Universities. Grant number ISP-Fysikk/216699 in Trondheim, Ås, and Oslo on research and education in computational physics

\end{enumerate}

\subsection*{Supervision of students and post-doctoral fellows}
During the last 21 years I have guided more than
100 graduate students (Master of Science and PhD levels) and post-doctoral
fellows. My full CV (see homepage above) contains a complete list of all students I have guided since 1999 in Europe and the USA.

\subsection*{Other relevant professional Experiences and Service last ten years}
\paragraph*{Editorial boards and committees.}
\begin{itemize}

\item Board member of the European Center for Theoretical Studies in Nuclear Physics and Related areas (ECT*), Trento, Italy, 2017-2020


\item Member of the Canadian research council's evaluation board on subatomic physics 2012-2015.

\item Member of the Swedish research council's evaluation board on subatomic physics 2007-2008.

\item Editorial Board member of Physical Review C (2014-2016)

\item Editorial Board member of European Physical Journal A (2010-2016)

\item Editorial Board member of European Physical Journal Special Topics (2010-present)

\item Editorial Board member of Springer's Lecture Notes in Physics (2010-present)

\item Editorial Board member of Springer's Undergraduate Lecture Notes in Physics (2014-present)

\item Editorial Board member of Springer's UniTexts in Physics (2016-present)

\item Editorial Board member of Springer's Graduate Texts in Physics (2017-present)

\item Editorial Board member of Springer's Undergraduate Texts in Physics (2016-present)

\item Editorial board member of Computers in Science and Discovery journal, a journal by IOP, UK (2008-2014)

\item \href{{http://fribtheoryalliance.org/}}{Steering Committee member of the FRIB theory alliance at Michigan State University (2013-2016)}

\item \href{{http://www.nucleartalent.org/}}{Initiated and led the Nuclear Talent initiative from 2010 till 2015, now member of the Steering committee}


\item January 2009-December 2011, leader of the Nuclear Physics group at the University of Oslo

\item Initiated in 2015 and chair of  the newly established Master of Science program in Computational Science, University of Oslo, a new crossdisciplinary
educational initiative with partners in physics, chemistry, biology, geoscience,
computer science, mathematics, statistics, and material science.

\item Member of the evaluation committee of the Institute of Nuclear Theory, University of Washington, Seattle, WA, USA, 2018

\end{itemize}

I am also the referee for more than 20 Scientific journals, spanning from Mathematical Physics to particle physics. In addition I have evaluated and evaluate research applications from 17 National Research Councils. I am also a member of more than ten International Advisory committees for various conferences.


\subsection*{Track record last ten years}

\paragraph*{Awards last ten years}

\begin{enumerate}
\item University of Oslo award for excellence in teaching for the \textbf{Computing in Science Education} project, 2011

\item NOKUT (Norwegian entity of quality assessment in higher education) award for excellence in teaching for the \textbf{Computing in Science Education} project, 2012

\item University of Oslo award for excellence in teaching for developing the Computational Physics group, 2015

\item Favorite graduate teacher at the Department of Physics and Astronomy at Michigan State University, 2016 

\item Olav Thon Foundation National prize for excellence in teaching award (National, all Norwegian higher education institutions, 2018

\item Award as best graduate teacher at the Department of Physics and Astronomy at Michigan State University, 2018

\item Merited teacher at the University of Oslo, 2020

\end{enumerate}


\paragraph*{Membership in Academies:}
\begin{itemize}
\item Elected member of the Norwegian Academy of Sciences and Letters, 2013

\item Elected member of the Royal Norwegian Society of Sciences and Letters, 2015 
\end{itemize}

\paragraph*{Books:}
\begin{enumerate}

\item M. Hjorth-Jensen, Maria Paola Lombardo, and Ubirajara Van Kolck (editors), \emph{Computational Nuclear Physics-Bridging the scales, from quarks to neutron stars}, Lectures Notes in Physics {\bf 936}, 2017.

\item Morten Hjorth-Jensen, \emph{Computational Physics, an introduction}, IOP, in press, 2020.

\item Morten Hjorth-Jensen, \emph{Computational Physics, an advanced course}, to be published by IOP in 2020.



\end{enumerate}

\paragraph*{Ten recent publications in journals with a referee system:}
\begin{enumerate}

\item Aynom T. Teweldebrhan, Thomas Schuler, John Burkhart, and Morten Hjorth-Jensen, Hydrology and Earth System Sciences {\bf 24}, (2020), 4641.

\item Fei Yuan, Sam Novario, Nathan Parzuchowski, Sarah Reimann, Scott K. Bogner and Morten Hjorth-Jensen,  Journal of Chemical Physics {\bf 147}, 164109 (2017).



\item G. Hagen, A. Ekstrom, C. Forssen , G. R. Jansen, W. Nazarewicz, T. Papenbrock, K. A. Wendt, S. Bacca, N. Barnea, B. Carlsson, C. Drischler, K. Hebeler, M. Hjorth-Jensen, M. Miorelli, G. Orlandini, A. Schwenk, and J. Simonis,Nature Physics {\bf 12}, 186 (2016).


\item T. Papenbrock, G. Hagen, M. Hjorth-Jensen, and  D. J. Dean, Reports on Progress in Physics {\bf  77}, 096302 (2014).

\item N. Tsunoda, K. Takayanagi, M. Hjorth-Jensen and T. Otsuka, Physical Review C {\bf 89}, :024313 (2014).

\item G. Baardsen, A. Ekstrom, G. Hagen, and M. Hjorth-Jensen,  Physical Review C {\bf  88}, 054312 (2013).  


\item A. Ekstrom, G. Baardsen, C. Forssen, G. Hagen, M. Hjorth-Jensen, G. R. Jansen, R. Machleidt, W. Nazarewicz, T. Papenbrock, J. Sarich, and S. M. Wild,  Physical Review Letters {\bf  110}, 192502 (2013).  


\item C. Forssen, G. Hagen, M. Hjorth-Jensen, W. Nazarewicz, and J. Rotureau, Physica Scripta {\bf T152}, 014022 (2013). 


\item Gaute Hagen, Morten Hjorth-Jensen, Gustav Ragnar Jansen, Ruprecht Machleidt, and Thomas Papenbrock, Physical Review Letters {\bf 109}, 032502, 2012. 

\item Gaute Hagen, Morten Hjorth-Jensen, Gustav Ragnar Jansen, Ruprecht Machleidt, and Thomas Papenbrock, Physical Review Letters {\bf 108}, 242501, 2012.  
\end{enumerate}

\end{document}

\paragraph*{Organization of workshops and schools during the last ten years}

I have organized more than 40 schools and workshops during the last ten years, the latest workshop was on
Qauntum Computing and Machine Learnig in Nuclear Physics, November 2-6, 2020, see \url{https://indico.ectstar.eu/event/85/} for meore details.




