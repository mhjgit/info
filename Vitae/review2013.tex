 \documentclass[prc,amsart,english,twocolumn,superscriptaddress,showpacs,floatfix]{revtex4}
 \usepackage[T1]{fontenc}       % DC-fonts
 \begin{document}


 \section*{Contribution from Morten Hjorth-Jensen (as of March 10 2013)}


 \subsection*{Highlights March 2012-March 2013}

We have studied stable and unstable 
nuclear system in the mass region from A=4 to A=60 and for the whole chain of isotopes such as the oxygen or the 
calcium isotopes. Our most recent works on the oxygen and calcium isotopes, with the inclusion of three-body forces and  
effects from the continuum, 
exhibit an unprecedented agreement with experiment in nuclear theory. Close to the driplines, our results also show also clear deviations from 
the shell-ordering proposed by the naive shell model. 
This has important consequences for our understanding of
 why matter is stable. Exploring the limits of stability of nuclear matter allows also for better constraints of popular
 mass model using in astrophysical modelling of the synthesis of the elements. 

Articles 9 and 10 below, both published in the Physical Review Letters, summarize the above.
Furthermore, we have recently performed, using the statistical software POUNDERS developed at Argonne National Laboratory, 
a better optimitization of nuclear forces in chiral perturbation theory. The results conveyed in article [1] below, using a better optimized nuclear force
for studies of properties of key nuclei and neutron matter, 
demonstrate that many aspects of nuclear structure can be understood in terms of this 
nucleon-nucleon interaction, without explicitly invoking three-nucleon forces.
This optimization program holds great promise for developing proper error estimates in nuclear many-body theory, with the potential to constrain
properly two- and three-nucleon and nuclear Hamiltonians for many mass regions.  

 \subsection*{Regular Articles in Journals with a Referee System only, after March 2012}
 \begin{enumerate}

 \item
A. Ekstr\"om, G. Baardsen, C. Forssen, G. Hagen, M. Hjorth-Jensen, G. R. Jansen, R. Machleidt, W. Nazarewicz, T. Papenbrock, J. Sarich, S. M. Wild, 
 \newblock {\bf  An optimized chiral nucleon-nucleon interaction at next-to-next-to-leading order}, 
 \newblock submitted to {\em Physical Review Letters},  and preprint arXiv:1303.4674.

 \item Lepailleur, A. and Sorlin, O. and Caceres, L. and Bastin, B. and Borcea, C. and Borcea, R. and Brown, B. A. and Gaudefroy, L. and Gr\'evy, S. and Grinyer, G. F. and Hagen, G. and Hjorth-Jensen, M. and Jansen, G. R. and Llidoo, O. and Negoita, F. and de Oliveira, F. and Porquet, M.-G. and Rotaru, F. and Saint-Laurent, M.-G. and Sohler, D. and Stanoiu, M. and Thomas, J. C.,
 \newblock {\bf Spectroscopy of $^{26}$F to Probe Proton-Neutron Forces Close to the Drip Line}, 
 \newblock {\em Physical Review Letters},  110:082502 (2013).

 \item D. D. DiJulio, J. Cederkall, C. Fahlander, A. Ekstr\"om, M. Hjorth-Jensen, M. Albers, V. Bildstein, A. Blazhev, I. Darby, T. Davinson, H. De Witte, J. Diriken, Ch. Fransen, K. Geibel, R. Gernhäuser, A. Görgen, H. Hess, K. Heyde, J. Iwanicki, R. Lutter, P. Reiter, M. Scheck, M. Seidlitz, S. Siem, J. Taprogge, G. M. Tveten, J. Van de Walle, D. Voulot, N. Warr, F. Wenander, and K. Wimmer
 \newblock {\bf Coulomb excitation of $^{107}$In}, 
 \newblock {\em Physical Review C},  87:017301 (2013).


 \item  C.~Forssen, G.~Hagen, M.~Hjorth-Jensen, W.~Nazarewicz, and J.~Rotureau,
 \newblock {\bf Living on the edge of stability, the limits of the nuclear landscape}, 
 \newblock {\em Physica Scripta},  T152:014022 (2013).

 \item Liddick, S. N. and Abromeit, B. and Ayres, A. and Bey, A. and Bingham, C. R. and Brown, B. A. and Cartegni, L. and Crawford, H. L. and Darby, I. G. and Grzywacz, R. and Ilyushkin, S. and Hjorth-Jensen, M. and Larson, N. and Madurga, M. and Miller, D. and Padgett, S. and Paulauskas, S. V. and Rajabali, M. M. and Rykaczewski, K. and Suchyta, S.
 \newblock {\bf Low-energy level schemes of ${}^{\mathbf{66},\mathbf{68}}$Fe and inferred proton and neutron excitations across $\mathbf{Z}=\mathbf{28}$ and $\mathbf{N}=\mathbf{40}$}, 
 \newblock {\em Physical Review C},  87:014325, 2013.


 \item D.~Dean, G.~Hagen, M.~Hjorth-Jensen, and T.~Papenbrock,
 \newblock {\bf Coupled-cluster computations of atomic nuclei}, 
 \newblock {\em Reports on Progress in Physics},  in press, 2013.


 \item D. D. DiJulio, J. Cederkall, C. Fahlander, A. Ekstr\"om, M. Hjorth-Jensen, M. Albers, V. Bildstein,
 A. Blazhev, I. Darby, T. Davinson, H. De Witte, J. Diriken, Ch. Fransen, K. Geibel, R. Gernh\"auser,
 A. G\"orgen, H. Hess, J. Iwanicki, R. Lutter, P. Reiter, M. Scheck, M. Seidlitz, S. Siem, J. Taprogge,
 G.M. Tveten, J. Van de Walle, D. Voulot, N. Warr, F. Wenander, and K. Wimmer,
 \newblock {\bf Excitation strengths in $^{109}$Sn: Single-neutron and collective excitations near $^{100}$Sn}, 
 \newblock {\em Physical Review C},  86:031302(R), 2012.

 \item D. D. DiJulio, J. Cederkall, C. Fahlander, A. Ekstr\"om, M. Hjorth-Jensen, M. Albers, V. Bildstein,
 A. Blazhev, I. Darby, T. Davinson, H. De Witte, J. Diriken, Ch. Fransen, K. Geibel, R. Gernh\"auser,
 A. G\"orgen, H. Hess, J. Iwanicki, R. Lutter, P. Reiter, M. Scheck, M. Seidlitz, S. Siem, J. Taprogge,
 G.M. Tveten, J. Van de Walle, D. Voulot, N. Warr, F. Wenander, and K. Wimmer,
 \newblock {\bf Coulomb excitation of $^{107}$Sn}, 
 \newblock {\em European Journal of Physics A}, 48:105,  2012.


 \item
 Gaute Hagen, Morten Hjorth-Jensen, Gustav Ragnar Jansen, Ruprecht Machleidt, and Thomas Papenbrock
 \newblock {\bf Evolution of shell structure in neutron-rich calcium isotopes }, 
 \newblock {\em Physical Review Letters}, 109:032502, 2012.

 \item
 Gaute Hagen, Morten Hjorth-Jensen, Gustav Ragnar Jansen, Ruprecht Machleidt, and Thomas Papenbrock
 \newblock {\bf Continuum effects and three-nucleon forces in neutron-rich
            oxygen isotopes}, 
 \newblock {\em Physical Review Letters}, 108:242501, 2012.

 \end{enumerate}


 \subsection*{Talks at workshops, conferences and institute seminars}
\begin{enumerate}
\item  {\em Computing in Science Education}. Seminar at college of engineering, 
College of Engineering, Michigan State University, East Lansing, Michigan, March 14 2012

\item  {\em Computing in Science Education, a new way to teach science?}, 
Institute colloquium at The Ohio State University, Department of Physics, 
Ohio State University, Columbus, Ohio, February 28 2012.

\item  {\em  Evolution of shell structure in neutron-rich isotopes}
Research seminar National Superconducting Cyclotron Laboratory, National Superconducting Cyclotron Laboratory, 
Michigan State University, East Lansing, Michigan, March 15 2012.

\item {\em  Evolution of shell structure in neutron-rich isotopes and the stability of nuclear matter}, 
Workshop on Exotic Nuclear Structure from Nucleons, Tokyo University, Tokyo, Japan, October 10-12 2012.

\item  {\em  Shell Structure in Neutron-rich isotopes and the stability of nuclear matter}, 
NSD Colloquia 2012, Lawrence Berkeley Laboratory, Berkeley, California, May 30 2012.

\item  {\em  Nuclear structure seminar at The Ohio State University}, 
Department of Physics, Ohio State University, Columbus, Ohio, February 29, 2012.

\item  {\em  Understanding the stability of nuclear matter},
Triangle Nuclear Theory Colloquium, University of North Carolina, Raleigh, North Carolina, May 1 2012.

\item  {\em 	Why is matter stable?}, Theory of Nuclear Physics Related to the RI Facilities, workshop at
Institute of Basic Sciences, Daejeon, Korea, May 11-12 2012.

\item {\em 	Why is matter stable? Understanding the limits of stability of nuclear matter}.
Nobel Symposium 152, conference, Gothenburg, June 10-15 2012.
\end{enumerate}


 \subsection*{Organization of Schools, conferences and member of scientific advisory boards}

\begin{enumerate}
\item Nuclear Talent school: High-performance computing and computational tools for nuclear physics, European
Center for Theoretical Studies in Nuclear Physics and Related areas, Trento, Italy, June 24-July 13, 2013.
I was the main organizer of the school and gave in addition also 15 lectures out of 45. I assisted also at the computer lab.
For more information, see \url{www.nucleartalent.org}, and go to course 9.

\item  Member of the International Advisory of Committee of the Zakopane conference on nuclear physics, extremes of the nuclear landscape, Zakopane, Poland, August 27-September 2, 2012. I was also a conveener on the topical session   "Modern approach to shell-model and beyond".

\item  I was also one of the organizers of the 'Third MSU-ORNL-University of Oslo winter school in Nuclear Physics', held at 
Oak Ridge National Laboratory, January 23-27, 2013. 

\item  I was also a member of the international advisory committee on ' International Conference on Mathematical 
Modeling in Physical Sciences', September 3-7, 2012, Budapest, Hungary.

\item  With Trygve Helgaker at the University of Oslo, I organized the  CMA-CTCC christmas workshop on
'Computational advances in many-body methods, from first principle methods to density functional theories', University of Oslo, 
December 19-20, 2012, Oslo, Norway.
\end{enumerate}

 \subsection*{Prizes, awards and professional recognitions}
\begin{enumerate}
\item 
The Norwegian Agency for Quality Assurance in Education (NOKUT) awarded their annual prize in 2012 
to the 'Computers in Science Education project'. The prize is shared with Hans Petter Langtangen, Anders Malthe-S\o renssen and Knut Morken (300.000 NOK).

\item March 22 2013 I became an elected fellow of the Norwegian Academy of Science and Letters

\end{enumerate}

\subsection*{Personal Summary}

I started last year (January 1 2012) at Michigan State University. I have a shared position between the University of Oslo and Michigan State University. 
In this summary I include what I have done till now at MSU, with further plans as well. 

Before I proceed, I should add  that these nine months (January-June 2012 and January-March 2013) spent at the NSCL@MSU have been fantastic scientifically.  
And teaching PHY981 is really fun, the students are excellent!

My main activity is on studies of nuclear physics systems, with an emphasis on many-body methods for nuclear structure studies. In particular, 
I wish to understand the stability of nuclear matter and nuclei from first principle methods. This matches perfectly the scientific mission of the NSCL and FRIB at MSU.  
Several of the articles published in the review period address these issues, amongts these, there are three Physical Review Letters article which study several 
nuclei close to the lines of stability.
Moreover, our recent optimizitation program holds great promise for the future. 
We have just started this work, but the results are extremely interesting.
We have recently optimized the nucleon-nucleon interaction from chiral effective field theory at next-to-next- to-leading order. The resulting new chiral force NNLOopt yields $\chi^2 \approx 1$ per degree of freedom for laboratory energies below approximately 125 MeV. In the $A = 3, 4$ nucleon systems, the 
contributions of three-nucleon forces are smaller than for previous parametrizations of chiral interactions. We have used this interaction 
to study properties of key nuclei and neutron matter, and find that many aspects of nuclear structure can be understood in terms of this 
nucleon-nucleon interaction, without explicitly invoking three-nucleon forces.

This optimization program opens up for a nuclear theory program whose outcome can lead to a serious test of chiral perturbation theory in a nuclear many-body environment,
with proper error quantifications in  nuclear many-body context. 
It is our goal to base a larger research application with several colleagues at the NSCL, Central Michigan  University, 
Oak Ridge National Laboratory and University of Tennessee, aiming at the above. Applying several many-body method and  developing a strong
computational axis, the aim is to develop optimized nuclear Hamiltonians that can be used to study properties of nuclei to be studied at FRIB. 

I have also a strong educational commitment, reflected in the initiation and partecipation in a project at the University of Oslo called 'Computing in Science Education'. This project has changed totally the way we teach science, with computations being introduced at first semester of study. Computations are included in all compulsory mathematics courses and almost all physics courses at the University of Oslo.  
The  Norwegian Agency for Quality Assurance in Education (NOKUT) awarded their annual prize in 2012 
to the 'Computing in Science Education project'. The prize is shared with Hans Petter Langtangen, Anders Malthe-Sorenssen and Knut Morken (300.000 NOK).
Together with Wolfgang Bauer and several colleagues at the College of Engineering, we are involved in implemeting something similar in the first
year of Engineering studies.

I am also involved in the so-called Nuclear Talent initiative, see \url{http://nucleartalent.org} for more details. 
Last year I organized the first Nuclear Talent school at the ECT* in Italy, with a total of 22 students. This year we will run two courses,
one on Nuclear Forces at the INT in Seattle, and one on Nuclear Reactions at Ganil. One NSCL student attended the course at the ECT* last year, and three more have been accepted at the INT course. In 2014 we are planning three courses, and most likely the Nuclear Astrophysics course will run at the NSCL.
 This initiative aims at providing and developing a broad curriculum in nuclear theory. An extended version of this project was attempted as an IGERT application from MSU this spring semester (a broader initiative in nuclear science in the USA, with Sean Liddick and Hendrik Schatz at the NSCL). We did however not pass the internal MSU qualification.  The criticism and feedback we received will be used to improve the Nuclear Talent concept, most likey as an activity under the auspices of a possible FRIB theory center, see below.

Finally, my long term goal is to be able to contribute to build up a strong activity on the nuclear many-body problem at MSU, an activity 
which will match the experimental program at the NSCL and FRIB. In the short term period, that is throughout 2014, parts of the start-up grant 
I received are used, together with the iCER project,
to hire a software/nuclear scientist  that can participate and aid the theory group in the development 
of new algorithms and methods for the nuclear many-body problem. 
Nicolas Michel began January 1 this year in this position. Furthermore, 
Andreas Ekstr\"om arrived here January 7 and will spend the rest of the year of 2013 as a post-doc at MSU with money from the Norwegian Research Council.
Andreas is central to the optimization program discussed above, and  together with us and 
colleagues  from Oak Ridge 
National Laboratory and Argonne National Laboratory (Mathematics), constructed a model for parametrizing (using all available two and three-body data)
a two-body and three-body interaction using effective field theory. This approach will therefore provide us with 
a consistent Hamiltonian for nuclear structure and reaction problems that includes three-body forces
as well. This Hamiltonian will then be used in the different many-body methods used by theorists at MSU, 
and will hopefully provided us with more consistent and predictive many-body approaches. 

Locally I have extensive collaborations with fellow theorists Scott Bogner and Alex Brown, and partly with Filomena Nunes. I collaborate closely with 
Sean Liddick and his students, and plan to start a collaboration with Alexandra Gade and some of her students on electromagnetic transitions in the tin isotopes.
Similarly, with Bill Lynch and Betty  Tsang, we have an ongoing discussion on spectroscopic factors. I hope also to get some graduate students soon.
There are two possible candidates, one local and one incoming from UArizona. 

On the longer term, I wish to contribute to the build-up of a strong theory center in nuclear physics, a theory
center which obviously is linked up with the needs and plans of FRIB. The first meeting for planning such a center is scheduled for April 2013. 




 \end{document}


