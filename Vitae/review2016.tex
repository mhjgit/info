 \documentclass[prc,amsart,english,twocolumn,superscriptaddress,showpacs,floatfix]{revtex4}
 \usepackage[T1]{fontenc}       % DC-fonts
 \begin{document}


 \section*{Contribution from Morten Hjorth-Jensen (May 2015-May 2016}

 \subsection*{Regular Articles in Journals with a Referee System only, after April 30 2015}
\begin{enumerate}
\item Erich W. Ormand, Alex B. Brown and Morten Hjorth-Jensen, \emph{First principles calculations for coefficients of the isobaric mass multiplet equation in the fp shell}, in preparation for \emph{Physical Review C}, 2016. 

\item Justin Lietz, Sam Novario, Gustav, Jansen, Gaute Hagen, and Morten Hjorth-Jensen, \emph{High-performance computing and infinite nuclear matter}, \emph{Lecture Notes in Physics}, in press, 2016.

\item Fei Yuan, Jørgen Høgberget, Titus Morris, Sam Novario, Nathan Parzuchowski, Sarah Reimann, Scott K. Bogner and Morten Hjorth-Jensen,   \emph{First principle calculations of quantum dot systems}, in preparation for Journal of Chemical Physics, 2016.

\item G. Hagen, M. Hjorth-Jensen, G. R. Jansen, T. Papenbrock, \emph{Emergent properties of nuclei from ab initio coupled-cluster calculations}, Focus issues of \emph{Physica Scripta}, 91:063006 (2016).

\item Naofumi Tsunoda, Takaharu Otsuka, Noritaka Shimizu, Morten Hjorth-Jensen, Kazuo Takayanagi, Toshio Suzuki, \emph{Exotic neutron-rich medium-mass nuclei with realistic nuclear forces}, \emph{Physical Review C}, in press

\item G. Hagen, A. Ekstrom, C. Forssen , G. R. Jansen, W. Nazarewicz, T. Papenbrock, K. A. Wendt, S. Bacca, N. Barnea, B. Carlsson, C. Drischler, K. Hebeler, M. Hjorth-Jensen, M. Miorelli, G. Orlandini, A. Schwenk, and J. Simonis,  \emph{Charge, neutron, and weak size of the atomic nucleus},  \emph{Nature Physics}, 12:186–190 (2016).

\item A. Ekstrom, G. R. Jansen, K. A. Wendt, G. Hagen, T. Papenbrock, B. D. Carlsson, C. Forssen, M. Hjorth-Jensen, P. Navratil, W. Nazarewicz,   \emph{Accurate nuclear radii and binding energies from a chiral interaction}, \emph{Physical Review C}, 91, 051301(R) (2015).

\item A. Ekstrom, B. D. Carlsson, K. A. Wendt, C. Forssén, M. Hjorth-Jensen, R. Machleidt, S. M. Wild,  \emph{Statistical uncertainties of a chiral interaction at next-to-next-to leading order},   \emph{Journal of Physics G}, 42:034003 (2015).



\item Osnes, E, Engeland, T, and Hjorth-Jensen, M, \emph{Large-scale shell-model study of Sn Isotopes}, European Journal of Physics Web of Conferences \textbf{95},01010 (2015)

\item Malthe-S\o renssen, Anders; Hjorth-Jensen, Morten; Langtangen, Hans Petter; M\o rken, Knut Martin. \emph{Integrasjon av beregninger i fysikkundervisningen}, UNIPED, 38:303, 2015. In Norwegian.
 \end{enumerate}


 \subsection*{Books}
My first book on Computational Physics was supposed to be published by the end of 2015, but I am delayed with the final text. I decided to rather finish the introductory and the advanced text as a whole. The two texts on Computational physics will be finalized during the fall this year. 
\begin{enumerate}
\item Morten Hjorth-Jensen, \emph{Computational Physics, an introduction}, to be published by IOP in 2016. Approx 400 pages

\item Morten Hjorth-Jensen, \emph{Computational Physics, an advanced course}, to be published by IOP in 2016. Approx 400 pages


\item M. Hjorth-Jensen, Maria Paola Lambardo, and Ubirajara Van Kolck (editors), \emph{Computational Nuclear Physics-Bridging the scales, from quarks to neutron stars}, to be published in Lectures Notes in Physics by Springer in 2016, 11 chapters with two contributions including MSU graduate students Justin Lietz, Titus Morris, Sam Novario, Nathan Parzuchowski and Fei Yuan. 
\end{enumerate}



 \subsection*{Talks at workshops, conferences and institute seminars and organization of meetings}
\begin{enumerate}
\item Hjorth-Jensen, Morten, {Computational Physics and Quantum Mechanical Systems}, one week course on Computational Physics at the University of Tunis El Manar, Tunis, Tunisia, May 16-20, 2016. In total 15 hours of lectures and 15 hours of computer lab and exercises. 

\item Hjorth-Jensen, Morten, {Correlations in many-body systems; from condensed matter physics to nuclear physics}, T-2, Nuclear and Particle Physics, Astrophysics and Cosmology, Los Alamos National Laboratory, New Mexico, USA, Tuesday, April 12, 2016

\item Hjorth-Jensen, Morten, {Integrating a Computational Perspective in the Basic Science Education}, Department of Physics Colloquium at Central  Michigan University,  Kalamazoo, Michigan, USA, April 4, 2016

\item Co-organizer with Giuseppina Orlandini and Alejandro Kievsky of Nuclear Talent course {Few-body methods and nuclear reactions}, ECT*, Trento, Italy, July 20-August 7 2015

\item Hjorth-Jensen, Morten, three lectures on nuclear structure theory at the National Nuclear Physics Summer School, Lake Tahoe, June 17-28 2015, California, 2015

\item Carlo Barbieri, Wim Dickhoff, Gaute Hagen, Morten Hjorth-Jensen, and Artur Polls, Nuclear Talent course on Many-body methods for nuclear physics, GANIL, Caen, France, July 5-25 2015. {Main organizer and teacher with in total five hours of lectures}. 

\item Hjorth-Jensen, Morten, ECT* {Doctoral Training Program 2015 on Computational Nuclear Physics}, April 13- May 22, ECT*, Trento, Italy. I taught the last week of the lecture series. In total I have ten one hour lectures. 

\end{enumerate}


 \subsection*{Service to the community}
\begin{itemize}
\item Editorial Board member of Physical Review C (2014-present)

\item Editorial Board member of European Physical Journal A (2010-present)

\item Editorial Board member of European Physical Journal Special Topics (2010-present)

\item Editorial Board member of Springer's Lecture Notes in Physics (2010-present)

\item Editorial Board member of Springer's Undergraduate Lecture Notes in Physics (2014-present)

\item {Steering Committee member of the FRIB theory alliance at Michigan State University (2013-2016)}

\item {Initiated and led the Nuclear Talent initiative from 2010 till 2015, now member of the Steering committee}

\item I initiated and lead the new {Master of Science program on Computational Science at the University of Oslo}. This is a new and multi-disciplinary program across several disciplines at the College of Natural Science of the University of Oslo. It includes now six departments at the University of Oslo.
\end{itemize}

\subsection*{Courses I teach}
During the last year I have been responsible for four courses, two at MSU (spring) and two at the University of Oslo (Fall semester)
\begin{itemize}
\item {FYS3150/4150 Computational Physics I}, Fall semester, senior undergraduate level (Oslo) 

\item {FYS4411 Computational Physics II: Quantum mechanical systems}, graduate level, Spring semester (Oslo) 

\item {PHYS981 Nuclear Structure}, graduate level, Spring semester (MSU) 

\item {PHY480/905 Computational Physics} (MSU), undergraduate and graduate level, Spring semester
\end{itemize}

I presently supervise 12 Master of Science students (University of Oslo) and four PhD students (MSU) either as main supervisor or co-supervisor. 

\subsection*{Lectures and organization of schools}
\begin{enumerate}

\item Hjorth-Jensen, Morten, {Computational Physics and Quantum Mechanical Systems}, one week course on Computational Physics at the University of Tunis El Manar, Tunis, Tunisia, May 16-20, 2016. In total 15 hours of lectures and 15 hours of computer lab and exercises. 

\item Co-organizer with Giuseppina Orlandini and Alejandro Kievsky of Nuclear Talent course {Few-body methods and nuclear reactions}, ECT*, Trento, Italy, July 20-August 7 2015

\item Carlo Barbieri, Wim Dickhoff, Gaute Hagen, Morten Hjorth-Jensen, and Artur Polls, Nuclear Talent course on Many-body methods for nuclear physics, GANIL, Caen, France, July 5-25 2015. {Main organizer and teacher with in total five hours of lectures}. 

\item Hjorth-Jensen, Morten, three lectures on nuclear structure theory at the National Nuclear Physics Summer School, Lake Tahoe, June 17-28 2015, California, 2015

\item Hjorth-Jensen, Morten, ECT* {Doctoral Training Program 2015 on Computational Nuclear Physics}, April 13- May 22, ECT*, Trento, Italy. I taught the last week of the lecture series. In total I gave ten one hour lectures. 
\end{enumerate}


 \subsection*{Prizes, awards and professional recognitions}
\begin{enumerate}
\item University of Oslo award for excellence in teaching for developing the Computational Physics group, 2015. This is an all university award with 250000 NOK (approx 40000 USD).

\item Favorite graduate teacher at the Department of Physics and Astronomy at Michigan State University, 2016 
\end{enumerate}

\subsection*{Personal Summary}

I started in January 2012 at Michigan State University. I have a
shared position between the University of Oslo and Michigan State
University.

My main activity is on studies of nuclear physics systems, with an
emphasis on many-body methods for nuclear structure studies. In
particular, I wish to understand the stability of nuclear matter and
nuclei from first principle methods. This matches perfectly the
scientific mission of the NSCL and FRIB at MSU.  Several of the
articles published in the review period address these issues, amongst
these there are several articles which study
nuclei close to the lines of stability.  Moreover, our recent
optimizitation program holds great promise for the future.  We 
started this work in 2012, and the results are extremely interesting.  We
have recently optimized the nucleon-nucleon and the three-nucleon interaction from chiral
effective field theory at next-to-next- to-leading order. These results will published soon in the Physical Review C. These Hamiltonians have then been applied to many other 
nuclear systems, a recent Nature Physics publication on calcium-48 and its neutron skin.

I have also a strong educational commitment, reflected in the initiation and partecipation in a project at the University of Oslo called 'Computing in Science Education'. This project has changed totally the way we teach science, with computations being introduced at first semester of study. Many of the seminars I have given at various US institutions deal with the integration of a computational approach to the basic science courses. Locally at MSU I am involved in teaching Computational Physics and have continuous discussions with several colleagues at the NSCL, the Department of Physics and Astronomy and the new Department of Computational Mathematics, Science and Engineering on computational issues and education in computational aspects. Much inspired by the developments at MSU, I have initiated and chair a Master of Science program in Computational Science at the University of Oslo, Norway. I spend the period July-December in Norway. 

I am also involved in the so-called Nuclear Talent initiative,  for more details. Last year I organized two of the courses and taught as well at one of the courses for three weeks. Next year with Alex Brown and Alexandra Gade, we are planning a Nuclear Talent course on Nuclear Theory for Nuclear structure experiments, most likely to be held at the ECT* in Italy. Last year I gave also lectures at the National Nuclear Physics summer school at Lake Tahoe in california. I taught this year also an intensive course on Computational Physics at the university of Tunis in Tunisia, with roughly 40 students attending. Last year I taught a week at the Doctoral training program of the ECT* in Italy. 

Locally I have extensive collaborations with fellow theorists Scott Bogner, Alex Brown, Heiko Hergert, Witek Nazarewicz and partly with Filomena Nunes, as well as tight collaborations with our graduate students and post-doctoral fellows. The Theory trailer is a lively community, with an average age well below 35. I collaborate also 
with many experimentalists at the lab. It is fun to be here!

Finally, my long term goal is to be able to contribute to build up a
strong activity on the nuclear many-body problem at MSU, an activity
which will match the experimental program at the NSCL and FRIB. With
Scott Bogner and Heiko Hergert we have now a group of several excellent graduate students. 


 \end{document}

